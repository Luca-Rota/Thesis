\chapter{Conclusions and Future Works} \label{ch:conclusions}

In conclusion, this thesis delves into the idea of \acrfull{ssi}, presenting an approach, to handling digital identities. It is evident that SSI shows 
potential in transforming how identities are controlled in the world, offering several benefits, over conventional centralized systems. Nevertheless, it is crucial to 
recognize that both, \gls{ssi} and the underlying blockchain technology, are still in their stages of development.

The thesis outlines a framework for managing \gls{ssi} throughout this exploration. By utilizing \acrfull{did}, \acrfull{vc} and blockchain technology this system delivers a decentralized 
solution for managing user identities. Designed as a decentralized browser application, it can be easily incorporated into various systems as an authentication tool.

Moreover, the integration of this framework into a real-world project has proven its usefulness, accompanied by the development of a user-friendly interface. However, it's 
important to note that given the stage of this field both the \gls{ssi} framework and the project its integrated into show areas that could be developed in the future. 

During the implementation phase of the \gls{ssi} framework identified limitations highlight two advancements that can support this application and similar ones; establishing 
standardized verifiable credentials issued by accredited entities and creating non-proprietary wallets capable of securely storing and managing these credentials for 
selective information disclosure. These changes would enhance system autonomy and reliability while boosting user privacy and security.

Regarding integration with the real project, future progress could focus on using for concrete purposes, the information contained within the VCs, released by the 
application. In addition, it is critical to improve the process of de-registering users from the application, to ensure proper management of the assets associated with users.

In summary, although this thesis has provided the groundwork for future work on identity management, through the creation and integration of an \gls{ssi} framework, there is a 
lot more that needs to be done to boost this emerging field. Recognizing the limitations identified and exploiting the research opportunities, will take us a step closer 
towards a secure and privacy preserving digital identity ecosystem. 
