\hypersetup{
    colorlinks=true,
    linkcolor=blue
}

\chapter{Introduction} \label{ch:intro}

In a world of constantly advancing technology, the very definition of identity has undergone a 
profound evolution, shifting from the traditional physical realm to the digital world. As a 
result, managing our identities now requires a reconsideration, leading to the exploration of 
Self-Sovereign Identity (SSI). 

Through the use of blockchain technology, this thesis delves into the world of decentralized 
identity, offering a new perspective on how we view, safeguard, and maintain control over our 
digital identities.

\section{Objectives} 

Through this project, set within the dynamic ecosystem of Links Foundation, our goal is to deeply 
investigate the concept of Self-Sovereign Identity (SSI). Our main objectives are twofold: first, 
to design an independent SSI framework for decentralized identity management using MetaMask and 
the Ethereum blockchain, and second, to seamlessly incorporate this framework into the advanced 
Data Cellar project at Links Foundation.

This journey is a deep exploration into the complexities of blockchain technology, tracing the 
origins of identity from its non-digital beginnings to its modern iteration. By delving into the 
cryptographic foundations of SSI, this study sheds light on the strong security and privacy 
measures in place, while also tackling the prevalent cybersecurity obstacles within the SSI 
realm. With practicality in mind, this research culminates in the creation of a robust and 
all-encompassing SSI framework, utilizing ground-breaking components such as MetaMask and a 
customized Ethereum smart contract to revolutionize decentralized identity management. 
The successful integration into Data Cellar is a testament to the adaptability and real-world 
significance of this newly developed framework.

At the end of this journey, lies the accomplishment of a decentralized application (DApp), a 
significant feat that not only showcases a sleek and intuitive interface crafted with React and 
JS for the frontend and NestJs for the backend, but also effortlessly integrates the SSI 
framework. This remarkable milestone not only tackles longstanding hurdles but also opens a 
gateway to a future where individuals have greater command over their digital identities.

\section{Outline}

Dopo aver spiegato nel Capitolo \hyperref[ch:intro]{[1]} gli obiettivi ed il lavoro prodotto per raggiungerli, il resto della
tesi è definito nel seguente modo:

\begin{itemize}
    \item Nella prima parte del Capitolo \hyperref[ch:verefoo]{[2]} si introduce il problema della Network Security Automation e si descrive il framework di Verefoo, ponendo particolare attenzione sul suo funzionamento ad alto e basso livello.
        Nella seconda parte sono descritte  le  definizioni delle Proprietà di sicurezza da passare come input al framework, con una spiegazione dettagliata
        di come queste intervengono nella definizione della topologia finale che verrà fornita come output dal framewrok. \\
        Infine verranno introdotti i grafi che verefoo richiede ed utilizza nella computazione dei vari \textit{NSF}.
    \item Il Capitolo \hyperref[ch:docker]{[3]} definisce l'architettura di docker, specificando la differenza tra usare docker per la virtualizzazione e delle semplici macchine virtuali. Successivamente
          viene fatto un approfondimento sul docker-compose, un tool in grado di poter istanziare più container velocemente tramite script. Nella parte finale viene spiegato come effettuare il networking
          sui container instanziati, come definirlo tramite docker-compose e come testare le comunicazioni in modo efficiente.
    \item Il Capitolo \hyperref[ch:ThesisObj]{[4]} descrive gli obiettivi posti all'inizio di questo lavoro di tesi. Più specificatamente, per ogni obiettivo presente verranno specificate le modalità e le scelte effettuate per portarlo a termine con una descrizione accurata dei vari passi che sono stati svolti prima della soluzione definitiva.
          Inoltre viene descritto in maniera più profonda rispetto a questo indice la descrizione dei futuri capitoli.
    \item Il Capitolo \hyperref[ch:ThesisObj]{[5]} descrive i lavori svolti nella prima delle due demo di cui questa tesi tratterà. Inizialmente viene descritto tramite pezzi di codice lo sviluppo dell'installer prodotto affinchè un qualsiasi utente possa
          utilizzare la demo in maniera pratica ed agile. Nei paragrafi successivi vengono evidenziati i punti critici incontrati, elencando le modifiche apportate affinchè essa possa funzionare correttamente.
          Nell'ultimo paragrafo infine verranno specificati ulteriori upgrade che si possono inserire nella demo per mettere in mostra in maniera ancora più evidente il lavoro svolto da Verefoo.
    \item Il Capitolo \hyperref[ch:intro]{[6]} descrive i lavori svolti ed implementati su Verefoo. In questo capitolo viene descritto il processo di merge fra le versioni precedentemente esistenti di Verefoo.
          Successivamente verrà quindi spiegato, anche tramite frammenti di codice, gli step
          che il framework eseguirà per produrre in output una rete che soddisfi contemporaneamente tutti i requisiti di sicurezza passati come input.  
          Infine si evidenziano anche le difficoltà che sono emerse lavorando al framework, e verranno proposte alcune soluzioni per poter evitare simili problematiche in futuro.
    \item Il Capitolo \hyperref[ch:intro]{[7]} descrive lo sviluppo della seconda Demo. In un primo momento viene mostrata la topologia
        di rete scelta da virtualizzare, con la finalità di indicare le nuove funzionalità di verefoo sviluppate al completamento del secondo obiettivo della tesi. Successivamente vengono descritti tutti i passi svolti per implementare la demo, con un commento per il codice che è stato utilizzato. Infine
        si evidenziano anche i limiti della demo prodotta con alcuni  futuri aggiornamenti possibili.
    \item Il Capitolo \hyperref[ch:conclusions]{[8]} elenca i lavori futuri da svolgere all'interno del framework, la necessità di poter implementare soluzioni
          alternative a quella proposta in questo documento, e i limiti che devono essere superati affinchè il framework possa essere utile in un ambiente reale e non
          solo di testing virtualizzato. Infine vengono descritte le conclusioni del lavoro, con un riassunto generale di tutto ciò che è stato prodotto.
\end{itemize}
