\chapter{Introduction} \label{ch:introduction}

In a world of constantly advancing technology, the concept of identity has experienced a significant transformation. It has shifted from being tied to our existence to 
encompassing the digital realm as well. Consequently the management of our identities now necessitates a reevaluation prompting us to explore the notion of \gls{ssi}. 
This thesis delves into the realm of identity using technology providing fresh insights, into how we perceive, protect and retain authority, over our digital identities.

\section{Objectives} 

This project delves into the ever-evolving ecosystem of Links Foundation to thoroughly explore the concept of \gls{ssi}. Our ambitious objectives 
consist of developing an autonomous \gls{ssi} framework for managing decentralized identities through the use of MetaMask and the Ethereum blockchain. Furthermore, we strive to 
seamlessly incorporate this cutting-edge framework into the advanced Data Cellar project at Links Foundation.

This exploration takes us into the intricacies of technology tracing the origins of identity from its non digital beginnings to its current form. By studying the foundations
of \gls{ssi} we gain insights, into the security and privacy measures in place as well as addressing the prevalent cybersecurity challenges within the \gls{ssi} domain.

With a focus on practicality this research culminates in developing a \gls{ssi} framework that incorporates groundbreaking elements like MetaMask and a customized Ethereum smart 
contract. This framework has the potential to revolutionize identity management as evidenced by its integration into Data Cellar.

As we reach the end of this journey we celebrate achieving an accomplishment – creating a \gls{dapp}. This achievement not showcases an user friendly interface 
built with React and JS for frontend development and NestJs for backend development but also seamlessly integrates our \gls{ssi} framework. This milestone not overcomes standing 
obstacles but also paves the way, for a future where individuals have greater control over their digital identities.

\section{Outline}

After a brief introduction and description of the goals of the thesis presented in Chapter 
\hyperref[ch:introduction]{[1]}, the remaining parts of the paper are structured as follows:

\begin{itemize}
    \item TODO
\end{itemize}


