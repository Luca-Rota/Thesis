\hypersetup{
    colorlinks=true,
    linkcolor=blue
}

\chapter{Introduction} \label{ch:introduction}

In a world of constantly advancing technology, the very definition of identity has undergone a 
profound evolution, shifting from the traditional physical realm to the digital world. As a 
result, managing our identities now requires a reconsideration, leading to the exploration of 
Self-Sovereign Identity (SSI). 

Through the use of blockchain technology, this thesis delves into the world of decentralized 
identity, offering a new perspective on how we view, safeguard, and maintain control over our 
digital identities.

\section{Objectives} 

Through this project, set within the dynamic ecosystem of Links Foundation, our goal is to deeply 
investigate the concept of Self-Sovereign Identity (SSI). Our main objectives are twofold: first, 
to design an independent SSI framework for decentralized identity management using MetaMask and 
the Ethereum blockchain, and second, to seamlessly incorporate this framework into the advanced 
Data Cellar project at Links Foundation.

This journey is a deep exploration into the complexities of blockchain technology, tracing the 
origins of identity from its non-digital beginnings to its modern iteration. By delving into the 
cryptographic foundations of SSI, this study sheds light on the strong security and privacy 
measures in place, while also tackling the prevalent cybersecurity obstacles within the SSI 
realm. With practicality in mind, this research culminates in the creation of a robust and 
all-encompassing SSI framework, utilizing ground-breaking components such as MetaMask and a 
customized Ethereum smart contract to revolutionize decentralized identity management. 
The successful integration into Data Cellar is a testament to the adaptability and real-world 
significance of this newly developed framework.

At the end of this journey, lies the accomplishment of a decentralized application (DApp), a 
significant feat that not only showcases a sleek and intuitive interface crafted with React and 
JS for the frontend and NestJs for the backend, but also effortlessly integrates the SSI 
framework. This remarkable milestone not only tackles longstanding hurdles but also opens a 
gateway to a future where individuals have greater command over their digital identities.

\section{Outline}

After a brief introduction and description of the goals of the thesis presented in Chapter 
\hyperref[ch:introduction]{[1]}, the remaining parts of the paper are structured as follows:

\begin{itemize}
    \item TODO
\end{itemize}


