\chapter{Introduction} \label{ch:introduction}

In a world of constantly advancing technology, the concept of identity has experienced significant transformation. Managing identity becomes a critical issue in this
time when our lives are becoming bound by digital platforms, services and, networks. Centralized identity systems, being popular, are however, encumbered by 
issues like the lack of user control and data breaches. As an answer to the above mentioned challenges, the idea of \acrfull{ssi} has been gaining more attention. \gls{ssi} 
adopts a decentralized and user-centric approach to identity management that enables individuals to assert and manage control over their identity data effectively. 
Through blockchain technology, \gls{ssi} enables a secure and trust-based architecture of identity verification, where intermediaries are not required and identity is 
protected from theft. The present research centers on the terrain of digital identity, especially concentrating on the standards and implementations of \gls{ssi}.

\section{Objectives} 

This thesis delves into the ever-evolving ecosystem of Links Foundation to thoroughly explore the concept of \gls{ssi}. Our ambitious objectives 
are to develop an autonomous \gls{ssi} framework for managing decentralized identities by using MetaMask and the Ethereum blockchain. Furthermore, we strive to 
seamlessly incorporate this cutting-edge framework into the advanced Data Cellar project at Links Foundation.

Link Foundation is an innovative organization that facilitates digital transformation via applied research, innovation, and technology transfer projects \cite{linksfoundation}.
Founded by the collaboration between the Compagnia di San Paolo and the Politecnico di Torino, the Links foundation is the central part of different technical and scientific 
disciplines, developing projects raging from Artificial Intelligence to Cybersecurity.

This exploration takes us into the intricacies of technology tracing the origins of identity from its non digital beginnings to its current form. By studying the foundations
of \gls{ssi} we gain insights, into the security and privacy measures in place as well as addressing the prevalent cybersecurity challenges within the \gls{ssi} domain.

With a focus on practicality this research culminates in developing a \gls{ssi} framework that incorporates groundbreaking elements like MetaMask and a customized Ethereum smart 
contract. This framework has the potential to revolutionize identity management, as evidenced by its integration into Data Cellar.

As we reach the end of this journey, we celebrate achieving an accomplishment, creating a \acrfull{dapp}. This achievement not showcases an user friendly interface 
built with React and JS for frontend development and NestJs for backend development but also seamlessly integrates our \gls{ssi} framework. This milestone not overcomes standing 
obstacles but also paves the way, for a future where individuals have greater control over their digital identities.

\section{Outline}

After a brief introduction and description of the goals of the thesis presented in Chapter \hyperref[ch:introduction]{[1]}, the remainder of the paper is structured as follows:

\begin{itemize}
  \item \textbf{Chapter \hyperref[ch:blockchain]{[2]}}: This chapter, therefore, is all about the background of the \gls{ssi} model used in 
  this study, which is blockchain technology. It starts with the definition and roles of blockchain, passes to the analysis of different types of blockchains, and ends 
  with a comparative review of two famous existing blockchains, namely Bitcoin and Ethereum.
  \item \textbf{Chapter \hyperref[ch:identity]{[3]}}: In this relatively concise chapter, an introduction to identity is provided, encompassing its evolutionary history 
  from the pre-digital era to a comparison of the three primary paradigms utilized for digital identity management: centralized , federated and user-centric.
  \item \textbf{Chapter \hyperref[ch:ssi]{[4]}}: This chapter focuses on the main theme of the thesis, namely the \gls{ssi} model. It evaluates the state-of-the-art of this 
  model of leading by starting from its historical context, then defining its main principles and presenting its advantages. Subsequently, the three main components of a 
  \gls{ssi} system, consist of \gls{did}, \gls{vc}, and Verifiable Data Registry, will be discussed 
  followed by an architecture analysis with a main focus on the "Trust Triangle" relation between the three principal entity: holder, verifier, and issuer.
  \item \textbf{Chapter \hyperref[ch:security]{[5]}}: According to the previous chapter's theme, this paragraph goes further to provide an in-depth study of the other 
  strong cryptographic and cybersecurity aspects of the \gls{ssi} model. Firstly, it highlights the working of VC proofs and then analyzes the attack vectors and their 
  vulnerabilities to which this model is prone.
  \item \textbf{Chapter \hyperref[ch:framework]{[6]}}: This chapter ends with the theoretical part and starts the practical discussion applied during the thesis period. 
  Overall, it explains the whole procedure of building an independent framework to handle the digital identities of individuals in a decentralized way under the \gls{ssi} model. 
  It demonstrates the instruments used for its implementation, the limitations encountered, and the final result.
  \item \textbf{Chapter \hyperref[ch:integration]{[7]}}: In this subsequent chapter, the integration of the previously mentioned framework into a practical project will be 
  presented to show its practical significance. It provides a brief background of the project and then proceeds to a comprehensive analysis before and after integration.
  \item \textbf{Chapter \hyperref[ch:conclusions]{[8]}}: This final chapter summarize the conclusions drawn from the research.It begins with a succinct summary outlining 
  the primary thesis topic and the executed practical endeavors, followed by an exposition of potential future work avenues necessitated by encountered limitations.
\end{itemize}