\chapter{Introduction} \label{ch:introduction}

In a world of constantly advancing technology, the concept of identity has experienced a significant transformation. Managing identity becomes a critical issue in the 
present time where our lives are becoming bound with digital platforms, services and networks. Centralized identity systems, being popular, is however, encumbered by 
issues like the lack of user control and the data breaches. As an answer to the above mentioned challenges, the idea of \acrfull{ssi} has been gaining more attention. \acrshort{ssi} 
adopts a decentralized and user-centric approach to identity management that enables the individual to effectively assert and manage control over their identity data. 
Through blockchain technology, \acrshort{ssi} enables a secure and trust-based architecture of identity verification, where intermediaries are not required and identity is 
protected from theft. The present research centers on the terrain of digital identity especially concentrating on the standards and implementations of \acrshort{ssi}.

\section{Objectives} 

This project delves into the ever-evolving ecosystem of Links Foundation to thoroughly explore the concept of \acrshort{ssi}. Our ambitious objectives 
consist of developing an autonomous \acrshort{ssi} framework for managing decentralized identities through the use of MetaMask and the Ethereum blockchain. Furthermore, we strive to 
seamlessly incorporate this cutting-edge framework into the advanced Data Cellar project at Links Foundation.

This exploration takes us into the intricacies of technology tracing the origins of identity from its non digital beginnings to its current form. By studying the foundations
of \acrshort{ssi} we gain insights, into the security and privacy measures in place as well as addressing the prevalent cybersecurity challenges within the \acrshort{ssi} domain.

With a focus on practicality this research culminates in developing a \acrshort{ssi} framework that incorporates groundbreaking elements like MetaMask and a customized Ethereum smart 
contract. This framework has the potential to revolutionize identity management as evidenced by its integration into Data Cellar.

As we reach the end of this journey we celebrate achieving an accomplishment, creating a \acrshort{dapp}. This achievement not showcases an user friendly interface 
built with React and JS for frontend development and NestJs for backend development but also seamlessly integrates our \acrshort{ssi} framework. This milestone not overcomes standing 
obstacles but also paves the way, for a future where individuals have greater control over their digital identities.

\section{Outline}

After a brief introduction and description of the goals of the thesis presented in Chapter 
\hyperref[ch:introduction]{[1]}, the remaining parts of the paper are structured as follows:

\begin{itemize}
    \item TODO
\end{itemize}


