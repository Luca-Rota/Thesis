\chapter{The Blockchain Technology} \label{ch:Blockchain}

In the current landscape, blockchain technology has attracted increasing global interest due to its promising applications and potential transformative impacts on various 
sectors. Founded in 2008 by Satoshi Nakamoto as the mainstay of the Bitcoin system \cite{9752154}, blockchain has evolved from a simple ledger of financial transactions 
to a fundamental technology that revolutionizes the way information is recorded, shared and managed within decentralized networks.

\section{What is the Blockchain}

The blockchain is a core technology that drives many decentralized systems offering transparency, trust and safety without having to have a central authority. It is 
basically a distributed ledger that operates through the peer-to-peer \cite{9596538}, \cite{ibm_blockchain}. This section talks of the key components and architecture in 
blockchain, revealing its fundamental features and operating principles.

\subsection{Core Elements of Blockchain}

The blockchain comprises several key elements essential to its functionality:

\begin{itemize}
  \item \textbf{Blocks}: A block is a basic unit comprising of transactional data. Every block is securely connected to the previous one; this connection creates a chain 
  of blocks. The transactions within a block are cryptographically secured; hence, immutability and integrity \cite{9596538}, \cite{ibm_blockchain}.
  \item \textbf{Transactions}: Transactions are diverse interactions in the blockchain network. These interactions are not limited to the financial transfers only; any 
  piece of valuable information can be considered as a transaction and diffused within the network \cite{9752154}, \cite{9036241}.
  \item \textbf{Decentralization}: Unlike centralized systems that depend on a single controlling authority, the blockchain is decentralized. It includes a web of 
  interdependent nodes, each replicating the distributed registry. Resilience, transparency and no single points of failure are guaranteed by decentralization \cite{9596538}, \cite{9752154}.
  \item \textbf{Consensus Mechanism}:  For verification and agreement of the state of a ledger, blockchain uses consensus mechanisms. The most common mechanism, \gls{pow}, 
  requires miners to solve the complicated cryptographic puzzles in order to add new blocks on the chain. Consensus mechanisms ensure consensus among network participants 
  to prevent attacks by malicious actors \cite{9596538}, \cite{9752154}.
\end{itemize}

\subsection{Blockchain Architecture}

The structure of the system is carefully crafted to uphold its principles of decentralization, immutability and transparency;

\begin{itemize}
  \item \textbf{\gls{dlt}}: At the core of blockchain lies DLT, where all participants, in the network have access to a ledger of transactions. 
  This shared ledger eliminates redundancy. Ensures that everyone has a trusted source of information across the network \cite{ibm_blockchain}, \cite{9752154}.
  \item \textbf{Immutable Records}: One key feature of blockchain is its records immutability. Once a transaction is recorded on the ledger it cannot be. Tampered with. 
  Any attempt to change a transaction requires adding an one preserving the integrity of data \cite{ibm_blockchain}.
  \item \textbf{Smart Contracts}: Smart contracts are self executing contracts with predefined rules embedded within the blockchain. These contracts. Enforce agreements 
  between parties speeding up transaction processing and reducing reliance on intermediaries \cite{ibm_blockchain}, \cite{9036241}.
  \item \textbf{Security Measures}:  Cryptography plays a role in ensuring security by protecting transactions from tampering and fraud. Private key pairs authenticate. 
  Enable secure digital identity management. Additionally cryptographic hashing guarantees data integrity and confidentiality \cite{9596538}, \cite{9036241}.
  \item \textbf{Peer-to-Peer Network}: The blockchain functions using a network architecture where participants can directly communicate and interact. This decentralized 
  structure promotes trust and resilience since transactions are verified and validated through consensus, among distributed nodes \cite{9752154}, \cite{9036241}.
\end{itemize}

\section{How the Blockchain works}

The function of a blockchain system encompasses a complex chain of procedures that safeguard the  integrity, transparency, and security of transactions. In this section, 
we delve into the fundamental mechanisms of blockchain.

\subsection{Transaction process}

\paragraph{Recording Transactions}

\begin{enumerate}
    \item \textbf{Starting a Transaction:} In a network, users initiate transactions through their wallets. Each wallet has a pair of keys. A key, for starting transactions 
    and a private key for authentication \cite{9596538}.
    \item \textbf{Creating Blocks:} As transactions take place they are grouped together into blocks of data. These blocks act as containers for recording details like 
    asset movement participants involved and timestamps \cite{ibm_blockchain}.
    \item \textbf{Hashing:} Every block contains information and refers to the hash of the previous block. Secure hash functions like SHA 256 play a role in ensuring the 
    integrity and immutability of transactions. Hashing allows each block to have a fingerprint making identification and verification simple \cite{ibm_blockchain} ,\cite{9036241}.
\end{enumerate}

\paragraph{Consensus Mechanism}

\begin{enumerate}
    \setcounter{enumi}{3}
    \item \textbf{Verification Process:} When a transaction is initiated the information is sent to a network of distributed peer, to peer nodes. These nodes work together 
    to confirm the validity of transactions using consensus mechanisms \cite{9036241}, \cite{geeksforgeeks}.
    \item \textbf{Formation of New Blocks:} Valid transactions are added to a \gls{mempool} where they wait to be included in a block. Miners, who are responsible 
    for creating blocks solve cryptographic puzzles in order to mine blocks. This process, known as \gls{pow} requires resources and time \cite{9596538}, \cite{geeksforgeeks}.
    \item \textbf{Consensus Algorithm:} In order to add a block to the blockchain nodes must agree on its validity through consensus algorithms. The miner who successfully 
    creates a block is rewarded. Consensus algorithms ensure that all nodes are synchronized and in agreement, about the state of the blockchain \cite{geeksforgeeks}.
    \item \textbf{Blockchain Integrity:} The interconnected nature of blocks ensures the immutability and integrity of the blockchain. Each block references the hash value 
    of its predecessor making it practically impossible to tamper with transactions without altering blocks \cite{9596538} ,\cite{9036241}.
\end{enumerate}

\subsection{Types of Consensus Mechanisms}

In the previous section we discussed an example of how blockchain uses a consensus mechanism called \gls{pow} \cite{9752154}, \cite{10037907}. However there are several 
types of consensus mechanisms that can be used in a blockchain network. 

In general, the two main types of consensus mechanisms are: \gls{pow} and \gls{pos}. With \gls{pow} miners solve puzzles to add new blocks, which requires significant 
computational resources and time \cite{9752154}, \cite{10037907}. This is elaborated further in Section \ref{subsec:bitcoin}. On the hand \gls{pos} assigns the right 
to create blocks based on participants cryptocurrency holdings providing benefits such, as energy efficiency and scalability advantages \cite{10037907}. Further information,
about PoS can be found in Section \ref{subsec:ethereum}.

There are also other consensus mechanisms beyond PoW and PoS; refer to the table for further details.

\section{Types and Advantages of Blockchains}

\section{Comparative of Bitcoin and Ethereum}

\subsection{Bitcoin} \label{subsec:bitcoin}

\subsection{Ethereum} \label{subsec:ethereum}